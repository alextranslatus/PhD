\documentclass[11pt]{article}
\usepackage[english]{babel}
\usepackage[top=.8in, bottom=1in, left=1.3in, right=1.3in]{geometry}
\usepackage[utf8x]{inputenc}
\usepackage{amsmath}
\usepackage{graphicx}
\usepackage[colorinlistoftodos]{todonotes}
\usepackage{enumitem}
\usepackage{listings}
\usepackage{filecontents}
\usepackage{verbatim}
\usepackage{textcomp}
\usepackage{multirow}
\usepackage{eurosym}
\usepackage{array}
\usepackage[export]{adjustbox}
\usepackage{dcolumn}
\usepackage{array}
\usepackage[authoryear,round]{natbib}
\renewcommand{\bibsection}{}
\bibliographystyle{abbrvnat}
\usepackage{lscape}
\usepackage{rotating}
\makeatletter
\newcommand{\ssymbol}[1]{^{\@fnsymbol{#1}}}
\makeatother
\usepackage[flushleft]{threeparttable}
\usepackage{setspace}
\usepackage{caption}
\usepackage{helvet}
\usepackage{float}
\usepackage{subcaption} % Optional: to customize float page (e.g. remove page number)
\usepackage{placeins}
\usepackage[hidelinks]{hyperref}
\usepackage{bibentry}
\nobibliography{biblio}
\usepackage{rotating}
\usepackage{booktabs}
\usepackage{makecell}
\usepackage{titlesec}
\usepackage{soul}
\usepackage{xcolor}

\sethlcolor{lime}
\newcommand{\hlyes}[1]{{\sethlcolor{lime}\hl{#1}}}
\newcommand{\hlno}[1]{{\sethlcolor{pink}\hl{#1}}}
\newcommand{\hlmaybe}[1]{{\sethlcolor{yellow}\hl{#1}}}


\titleformat{\section}
{\fontsize{16}{18}\bfseries\sffamily}
{\thesection}{1em}{}

\titleformat{\subsection}
{\fontsize{14}{18}\bfseries\sffamily}
{\thesubsection}{1em}{}

\titleformat{\subsubsection}
{\fontsize{13}{18}\bfseries\sffamily}
{\thesubsubsection}{1em}{}

\titleformat{\subsubsubsection}
{\fontsize{11}{18}\bfseries\sffamily\itshape}
{\thesubsubsubsection}{1em}{}


\title{Feedback from Nicole on EDSD project}
\date{September 2025}

\begin{document}
	
	\maketitle
	
	In this document, you'll find a summary of the feedback I got, to which I've added my impressions through highlighting and comments, using the following key:
	
	\noindent \hlyes{Yes please}
	
	\noindent \hlno{Don't like the sound of that}
	
	\noindent \hlmaybe{Not sure I understand}

	\noindent \textbf{In bold text, comment by Alex}
	
	In section \ref{Moving on}, I suggest possible changes.
	
	
\section{Strengths}


\begin{itemize}
	\item Uses two high-quality, large-scale, nationally representative cohort studies.
	
	\item Clear conceptualization of three parenting dimensions.
	
	\item Novelty: systematically documents both gender and class differences in early childhood parenting across two institutional contexts.
	
	\item Transparent discussion of comparability limitations and attrition.
	
\end{itemize}


\section{Limitations}
	
\begin{itemize}
\item Measurement comparability: The operationalization of expectations, activities, and division of labour differs substantially across datasets (e.g., ranked wishes in Elfe vs. binary “yes/no” in MCS). This weakens cross-country comparisons. \textbf{Yes.}

\item Father’s role: Analyses largely rely on mothers’ reports; limits insights into fathers’ perspectives and father–child interactions. \textbf{Indeed, for this paper, with this data, fathers aren't the focus/are left out quite a bit.}

\item Causal interpretation: Results are descriptive associations; the discussion sometimes implies mechanisms (e.g., “same-sex dyads”) without testing them. \textbf{Agreed. What if I cite the literature to back up these ideas of interpretation in the discussion? Does that work?}

\item \hlyes{Scope of class measure: Mother’s education is a reasonable proxy, but social class is broader (income, occupation, cultural capital). The thesis would benefit from sensitivity checks.}

\item \hlmaybe{Theoretical depth: While “doing gender” and Lareau’s “concerted cultivation” vs. “natural growth” are invoked, engagement with mechanisms (e.g., how institutional contexts interact with family practices) could be deepened.}

\item Findings on girls’ advantage: \hlyes{The surprising result that girls get more access and exposure to family resources deserves a stronger theoretical explanation}. Could this be survey artefacts, mother-report bias, or genuine shifts in gender socialization?

\end{itemize}


\section{Questions}

\begin{itemize}
	\item Conceptual / Theoretical
	
	Why did you choose “doing gender” as the main theoretical lens, and how does it help explain your findings compared to \hlmaybe{alternative frameworks (e.g., gendered investment theory, social learning)}?
	
	How do you reconcile the surprising finding that girls seem to receive more resources, given prior literature showing parental bias toward boys?
	
	\item Data \& Measurement
	
	What are the implications of measuring “social class” only through mother’s education? Would results differ if you used income or occupation?
	
	Given that most parenting measures are reported by mothers, how might father-reported data change your results?
	
	How comparable are the Elfe and MCS measures across the three parenting dimensions? To what extent can we draw cross-country conclusions?
	
	\item Methods
	
	\hlyes{Why did you choose linear probability models over logistic regression? What are the trade-offs, and do they affect your substantive conclusions?}
	
	You mention using Bonferroni corrections. Did you consider the risk of over-correction and missing real effects?
	 
	\item Findings
	
	In both countries, gender differences were weaker than class differences. What does this imply for policies aiming to reduce gender inequality through early childhood interventions?
	
	You found no strong evidence of interaction between gender and social class. How do you interpret this—do parents reproduce gender and class inequalities in parallel rather than in intersectional ways?
	
	\hlyes{Could the finding that UK mothers take on more domestic/childcare work than French mothers be partly due to part-time work arrangements?}
	
	\item Limitations / Future Work
	
	\hlyes{What are the main blind spots of your study, and how would you design future research to address them?}
	
	\hlyes{If you could redesign the data collection in these cohort studies, what would you add or change to better capture gendered parenting?}
	
	
\end{itemize}

	
	\newpage 
	
	\section{Moving on} \label{Moving on}
	
	\begin{itemize}
		\item I'm not convinced by the comparative approach myself.
		\item I need help with respect to interpretations, how far can I take it? Can I do it citing the literature? For example,  how can I discuss my findings on girls’ advantage for activities? Survey artefacts, mother-report bias, or genuine shifts in gender socialization? Do I put them as questions? And look into the literature to see which might be more likely?
		\item Social class/education: figure out which one.
		\item Look into gendered investment theory and social learning.
	\end{itemize}
	
	
\end{document}