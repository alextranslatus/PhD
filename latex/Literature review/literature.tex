\documentclass[11pt]{article}
\usepackage[english]{babel}
\usepackage[top=.8in, bottom=1in, left=1.3in, right=1.3in]{geometry}
\usepackage[utf8x]{inputenc}
\usepackage{amsmath}
\usepackage{graphicx}
\usepackage[colorinlistoftodos]{todonotes}
\usepackage{enumitem}
\usepackage{listings}
\usepackage{filecontents}
\usepackage{verbatim}
\usepackage{textcomp}
\usepackage{multirow}
\usepackage{eurosym}
\usepackage{array}
\usepackage[export]{adjustbox}
\usepackage{dcolumn}
\usepackage{array}
\usepackage[authoryear,round]{natbib}
\renewcommand{\bibsection}{}
\bibliographystyle{abbrvnat}
\usepackage{lscape}
\usepackage{rotating}
\makeatletter
\newcommand{\ssymbol}[1]{^{\@fnsymbol{#1}}}
\makeatother
\usepackage[flushleft]{threeparttable}
\usepackage{setspace}
\usepackage{caption}
\usepackage{helvet}
\usepackage{float}
\usepackage{subcaption} % Optional: to customize float page (e.g. remove page number)
\usepackage{placeins}
\usepackage[hidelinks]{hyperref}
\usepackage{bibentry}
\nobibliography{biblio}
\usepackage{rotating}
\usepackage{booktabs}
\usepackage{makecell}
\usepackage{titlesec}

\titleformat{\section}
{\fontsize{16}{18}\bfseries\sffamily}
{\thesection}{1em}{}

\titleformat{\subsection}
{\fontsize{14}{18}\bfseries\sffamily}
{\thesubsection}{1em}{}

\titleformat{\subsubsection}
{\fontsize{13}{18}\bfseries\sffamily}
{\thesubsubsection}{1em}{}

\titleformat{\subsubsubsection}
{\fontsize{11}{18}\bfseries\sffamily\itshape}
{\thesubsubsubsection}{1em}{}


\title{Notes on reading}
\date{}

\begin{document}
	
\maketitle

\tableofcontents

\newpage



\section{Reading synthesis}

Why I read these papers and what I got from them
\\
\\
\noindent \textbf{\bibentry{mesman_gendered_2018} }

I read it because it focuses on early childhood, and it discusses different ways of conceptualising parenting. While not being a meta-analysis, this paper explains that researchers conclude that gendered parenting is uncommon nowadays because the literature shows few differences in parenting styles (with biggest but small differences found in early childhood) and explicit measures \citep{darling_parenting_1993, lytton_parents_1991}, and because the theory on parents’ role in fostering gendered child development focus on specific parenting practices \citep{martin_cognitive_2002}. The authors explain that it's not surprising we find few differences in explicit measures, because of social desirability bias in societies that value gender equality \citep{axinn_gender_2011}.

So instead, the authors suggest we should also pay attention to implicit measures: “implicit gendered parenting practices are covert behaviors and statements by parents that convey messages about differential expectations of girls and boys without stating these messages overtly”. There are direct implicit messages that concern the child and their behaviour. The authors argue that the highly gendered stereotyping of popular child products \citep{murnen_boys_2016} is a manifestation of parents choosing to expose their children to things that convey gendered messages, even when they don't endorse these messages explicitely. The authors also cite evidence on mothers of boys discouraging risky and disruptive behaviour less \citep{martin2005, morrongiello_mothers_2000, power_patterns_1986} and encouraging prosocial behaviours less \citep{ross_maternal_1990, smetana_toddlers_1989}.

There are also indirect implicit messages that concern others and that reach the child vicariously (vicarious social learning theory: children pick up on gendered evaluative messages regarding others’ actions \citep{bussey_social_1999}). A host of papers describe getting parents to read picture books to elicit gender-relevant talk, with results showing less explicit messages by parents, and more implicit messages via differential evaluating and labeling of stereotypical and counterstereotypical behaviors \citep{deloache_three_1987, friedman_mothers_2007, gelman_mother-child_2004, endendijk_boys_2014, van_der_pol_fathers_2015}. The authors also argue that modeling is a form of indirect gendered message as it gives children information about gender roles, providing them with guidelines for behaviour in similar situations \citep{bussey_social_1999}. In line with this idea, the authors cite evidence on children from families with more traditional gender roles having more gender stereotypical expectations \citep{sinno_moms_2009}. "

Lastly, the authors offer directions for future research, including: how are explicit and implicit messages intertwined? To what extent do different dimensions matter in shaping scientific conclusions on these topics; what motives drive parents? (Do they make differences to socialise their children in a way that prepares them
for adult gender roles and society’s gendered expectations, while not wanting to reveal this goal if they believe it contradicts society’s ideology of gender equality? or are they victims of unconscious gender stereotypes pervasive in many societies \citep{miller_womens_2015}?) What happens when societies undergo various evolutions with respect to gender roles? What about non-Western contexts? What about in different socioeconomic backgrounds? What child outcomes does gendered parenting produce? (It could prepare them for socially adaptive functioning this promoting greater well-being, or it could waste talent and force people into lifestyles and careers that deny personal identities.)

I wanted more depth and texture, so I turned to the qualitative paper by \cite{kane2006no}: 
\\
\\
\noindent \textbf{\bibentry{kane2006no} }

She interviewed parents and noted whether they made positive or negative comments about their child's conformity or non-conformity with respect to gender. She interviewed 24 mothers and 18 fathers of three- to five-year-olds in 1999-2002. Parents had diverse family types, class, race, sexual orientations, education. The main finding is that parents of boys made both negative and ecouraging comments when their child acted in nonconform ways, while parents of girls mostly welcomed gender nonconformity in their child.

The author takes a special interest in the reactions to boys' nonconfirmity, but before moving on to that, I thought this was interesting with regard to girls' nonconfirmity: Though the comments about girls' gender nonconformity were more positive, the author reminds us that it's possible that negative responses from parents to perceived departures from traditional femininity would be more notable as girls reach adolescence, citing \cite{pipher1998}: "That’s when the gender roles get set in cement, and that’s when girls need tremendous support in resisting cultural definitions of femininity."

By explaining with greater detail why parents had more complex reactions to boys' nonconfirmity, she builds on three strands of literature: we have research on gendered parenting which she says is mostly quantitative and doesn't go into the reasons for and awareness of gendered parenting; with the doing gender theory, we can think of parents as actors involved in a more complex process of accomplishing gender with and for their children, either in accordance with the expectations of others but also when they resist or stray from such expectations \citep{fenstermaker2002genderinequality}; and these expectations are those of normative conceptions of masculinity, i.e. hegemonic masculinity, which include aggression, limited emotionality, and heterosexuality, with the aim of legitimating male domination \citep{connell1987}.

The interview questions mainly focused on the current activities, toys, clothes, behaviors, and gender awareness of the focal child and the parents’ perceptions of the origins of these outcomes, as well as their feelings about their children’s behaviours and characteristics in relation to gendered expectations.

Parents of boys, especially mothers, were generally fairly encouraging of their child acquiring domestic abilities and an orientation toward nurturance and empathy. These were considered as nontraditional but positive for boys. However, anything related to icons of feminity (wearing pink or frilly clothing; wearing skirts, dresses, or tights; and playing dress up in any kind of feminine attire, nail polish, dance, especially ballet, Barbie dolls) got more negative reactions. Some parents described taking action to steer their child away from them, or to find compromises with their child. Some parents expressed concern that boys might develop too much emotionality and passivity as a consequence of these exposures, rather than learn how to fight for what they want. Parents also expressed concern that their boy would become gay, or be perceived as such. Heterosexual fathers most often saw this possibility as a threat to their own masculinity, while mothers and gay parents were most concerned about accountability to gender assessment by others (including by the father of their child).
\\
\\
\noindent \textbf{\bibentry{lytton_parents_1991} }

The paper by \cite{mesman_gendered_2018} mentioned this meta-analysis, to say that there aren't differences in parenting by child sex. And this is true for most areas the authors explored: amount of interaction, achievement encouragement, warmth, encouragement of dependency, discouragement of aggression, use of reasoning. But they did find differences for disciplinary methods and encouragement of sex-types activities/perceptions, with bigger effects in early childhood and for fathers rather than for mothers.
\\
\\
\noindent \textbf{\bibentry{morawska_effects_2020} }

And then there's this systematic review that shows differences in parenting in early childhood: they vocalised differently, used different socialising strategies, played differently and provided different toys to their sons and daughters. They find that this differential parenting was associated with some differences in child development across child gender: including differences in child vocalisation, displays of affect, pain responses, compliance, toy play and aggression.

However, this review isn't based on many papers (45 studies, of which 14 focus on vocalisation, 21 on socialisation, 7 on play, 3 on toys). Authors also point to the low quality of these studies: lack of longitudinal studies.

The authors of this review explain that theories commonly used to explain gender development have limited focus on the substantial influence of parents in the early lives of children: sociological \citep{connell2015}, evolutionary \citep{archer1996}, cognitive-developmental \citep{kohlberg1966}, gender  schema \citep{bem1981, martin1981} and social constructionist \citep{fine2015} theories. They argue that the social cognitive theory \citep{bussey_social_1999} is better suited, as it it highlights the role of parents. It proposes that gender role behaviour arises as a result of multiple reciprocal interactions between the child’s individual variables and both the family context and the broader social-cultural system.

The authors argue that the first years of life have long been recognised as fundamental to future development, and the role of parents is widely acknowledged as being critical. They cite evidence of gendered differences in the early years of a child's life: pregnant mothers describe their foetus’ movement differently when they know their baby’s gender \citep{rothman1986}; 2-year-olds recognise which toys are for boys versus girls \citep{serbin2001}; preschool boys are constrained to playing with masculine toys when engaging with adults \citep{wood2002}; and 6-year old girls are less likely than boys to believe that members of their own gender are “really, really smart” \citep{bian2017}.
\\
\\
\noindent \textbf{\bibentry{culp1983comparison} }

Reading these papers about differences in parenting in early childhood, I was wondering about whether children already showing differences in their behaviours that could influence their parents' reactions. This paper by \cite{culp1983comparison} is quite old and very succint (lacks many details such as a description of their sample), but it describes an experiment implemented that showed that parents of infants reacted differently with actor infants, whether the infant was dressed as a boy or as a girl. This paper suggests that even when parents aren't aware of it, they do act differently around boys and girls even in the absence of difference in the behaviour and characteristics of the child. Though I find this convincing for 6-months old infants, I'm less convinced that exterior influences can be excluded starting around age 2, when children start displaying gendered behaviours. How do we exclude this mechanism?
\\
\\
\noindent \textbf{\bibentry{verhoeven_parenting_2010} }

It's also for that reason that I read this empirical paper on boys' aggression, because of its focus on working out the direction of causality. This paper took advantage of repeated measures over time. The authors conducted their interviews at multiple time points. They found that it was boys' behaviour that predicted their parents' behaviour, rather than the other way round. However, the sample was not representative of a general population as it focused on mostly white and educated families.

This paper exposes nicely the literature on bidirectional effects between children and adults, citing papers that look at the role of the child as an actor in the socialisation.

It also touches on the idea that some effects might not be measurable yet: the significance of parenting behavior for children’s externalizing behaviors may not be evident before children enter school.
\\
\\
\noindent \textbf{\bibentry{} }



\section{Literature overviews: meta-analyses, systematic reviews, perspectives}

\subsection{The effects of gendered parenting on child development outcomes: A systematic review \citep{morawska_effects_2020}}

\subsubsection*{Nutshell}

Systematic review, finds that parents do respond differently to their children depending on their gender: vocalised differently, used different socialising strategies, played differently and provided different toys. Founds associations with some child outcomes: child vocalisation, displays of affect, pain responses, compliance, toy play and aggression. But overall low quality of evidence, lack of longitudinal studies: need a more systematic approach to reach better conclusions

\subsubsection*{Main ideas}

Gender stereotypes have not changed significantly over time \citep{haines2016}, including in more recent years \citep{macphee2019, miller2018}.

A number of theoretical approaches, including sociological \citep{connell2015}, evolutionary \citep{archer1996}, cognitive-developmental \citep{kohlberg1966}, gender schema \citep{bem1981, martin1981} and social constructionist \citep{fine2015}, have been used to describe the process of gender role development.

Social cognitive theory \citep{bussey1984, bussey_social_1999} highlights the role of parents, provides a model of the earliest processes of gender role construction, and supports the development of testable hypotheses about the development of gender roles.

While there is little evidence that parents use different broad styles of parenting with boys and girls \citep{endendijk2016gender, lytton_parents_1991}, this is not surprising given most socialisation tasks are relevant to all children. In societies that value gender equity, most parents are also not likely to explicitly report differences in parenting boys versus girls \citep{mesman_gendered_2018}. Gendered parenting is more likely to be expressed in specific parenting practices, influenced by parental beliefs and attitudes \citep{endendijk2013}.

more recent research demonstrates that gender socialisation is related to parents’ gender stereotypes rather than to child behaviour \citep{endendijk2017}.

also some indication that younger children receive more sex role socialisation than older children \citep{fagot1991, lytton_parents_1991}, consistent with a social-learning model of gender role acquisition. Intergenerational transmission of gender role attitudes explains a substantive part of gender inequalities \citep{johnston2014}.

Although a gender-neutral approach to parenting has been suggested by some authors, there is limited evidence to support the claim that a gender-neutral approach could decrease gender-typed behaviour \citep{martin2005william, nguyen2008}.

The focus on parents of younger children was deliberate, as this is the period when gender role stereotypes develop and parental effects are most likely to be evident.

Any outcome measure of child development (e.g., language, school readiness), behaviour (e.g., aggression, prosocial behaviour), emotion (e.g., anxiety) or health (e.g., physical activity, overweight) was included.

For the purposes of this review, the focus is on those outcomes within studies where gender-differentiated parenting was found, acknowledging that there were also many null findings.

the inclusion criteria for this study required the presence of both gender-differentiated parenting and measurement of child outcomes.

In terms of gendered parenting, while there were many areas where parents did not differ in their parenting towards boys and girls, across each of the domains there were also a number of significant differences.

So do these differences in parenting lead to different outcomes in children’s development? While there was some limited evidence of the effect of gendered parenting on children’s development, it is critical to note that many studies were correlational and the individual child outcomes were in most cases not specifically linked in the study to particular parenting behaviours. Thus, while the majority of studies (n = 35, 77.8\%) did find differences in child development outcomes, it is noteworthy that not only was there considerable diversity and heterogeneity in the outcomes assessed, but the association between parenting and child outcomes was not directly assessed in any of the correlational studies. Likewise, while the majority of the longitudinal studies (11/12) showed similar gender differences in child outcomes, their connections to specific parenting behaviours was not clearly tested.

clearly more research to establish the link between parental behaviour and child development is needed. For example, how does gendered parenting during play affect the child’s development? We can speculate, for example, that as parents provide different toys to children, and subsequently encourage same-sex-stereotyped play, they socialise children to different roles and thus encourage the development of different skills (e.g., nurturing via doll play in girls).

Despite the focus on parenting behaviour, we also need to consider parental perceptions, attitudes, attributions and beliefs that may influence parental behaviour, in addition to broader societal factors including the media. These factors are likely to influence parental behaviours both directly and indirectly, and are therefore important to understand in providing a framework for the transmission of gender role stereotypes.

it is also relevant to examine the extent to which child behaviours and preferences shape parental responses.

In this study, the clearest finding was the existence of gender-differentiated parenting behaviour. Consistent with the theory, children were reinforced for gender stereotypical play, had access to gender stereotypical toys and received discouragement for counter-stereotypical behaviours.

there was less evidence for the effects of these behaviours on children, often because parent behaviours were not explicitly linked to child outcomes, and reciprocal interactions between parents and children were rarely examined.

\cite{endendijk2018}'s Gendered Family Process Model

The conclusions of any systematic reviews are dependent on the quality of the included studies.

The question of whether boys and girls within the same family are parented differently has not been addressed in the literature, as noted by previous reviews \citep{endendijk2016gender}.

\subsection{Parents' differential socialization of boys and girls: A meta-analysis. \citep{lytton_parents_1991}}

\subsubsection*{Nutshell}

Meta-analysis of 172 studies, on whether parents make systematic differences in their rearing of boys and girls

Most effects were small and not significant, except for physical punishment and encouragement of sex-typed activities.

Fathers differentiate more.

Effects decrease with age.

"Because little differential socialization for social behavior or abilities can be found, other factors that may explain the genesis of documented sex differences are discussed."

\begin{figure}[H]
	\centering
	\noindent\includegraphics[width=\textwidth]{/Users/alexsheridan/Documents/Work/PhD/latex/figures/lytton1991fig.png}
\end{figure}


\subsubsection*{Check out}

\bibentry{glass1981}

\bibentry{kohlberg1966}

\subsubsection*{Main ideas: to complete}

\subsection{Gendered Parenting in Early Childhood: Subtle But Unmistakable if You Know Where to Look \citep{mesman_gendered_2018}}

\subsubsection*{Nutshell}

Says no evidence that parents treat boys and girls differently explicitly, but evidence they do implicitly

Limitation: this isn't a meta-analysis

\begin{figure}[H]
	\centering
	\noindent\includegraphics[width=\textwidth]{/Users/alexsheridan/Documents/Work/PhD/latex/figures/mesman2018fig.png}
\end{figure}

\subsubsection*{Check out}

\textit{\bibentry{lytton_parents_1991}}

\medskip

\noindent \textit{\bibentry{verhoeven_parenting_2010}}

\medskip

\noindent \textit{\bibentry{culp1983comparison}}

\medskip

\noindent \textit{\bibentry{bussey_social_1999}}

\medskip

\noindent \bibentry{martin_cognitive_2002}

\medskip

\noindent \bibentry{else-quest_gender_2006}


\subsubsection*{Main ideas}

Similar parenting styles with sons and daughters \citep{darling_parenting_1993, lytton_parents_1991}, effect sizes decreased with age, started out small

Theories on parents’ role in fostering gendered child development focus on specific parenting practices \citep{martin_cognitive_2002}

observational research has taught us that explicit messages to children (e.g., dolls are for girls) are rare in societies that value gender equality \citep{nosek_harvesting_2002}; and parents are reluctant to report gender-stereotypical ideas because they think they would be frowned upon \citep{axinn_gender_2011}

lack of evidence on differences in parenting styles or explicit gendered parenting practices leads researchers to conclude that gendered parenting is uncommon nowadays \citep{lytton_parents_1991}

But what about implicit gendered parenting:

“Implicit gendered parenting practices are covert behaviors and statements by parents that convey messages about differential expectations of girls and boys without stating these messages overtly” which “can be divided further into direct and indirect messages. Direct messages concern the child and his or her behaviors, skills, and interests. Indirect messages convey information that concerns others or reflects general observations regarding gender that reach the child vicariously.”

Direct messages: “parenting choices such as the films, books, and commercial products to which they expose their children and that convey gendered messages even if the parents do not endorse such messages explicitly” parenting control most of this input in early childhood

Evidence: popular commercial child products are highly gender stereotyped \citep{murnen_boys_2016}; and this exposure fosters children’s gendered cognitions and behaviours \citep{coyne_its_2014}

More evidence: mothers respond less negatively to a son’s risky and disruptive behaviors \citep{martin2005william, morrongiello_mothers_2000, power_patterns_1986} and are less encouraging of a son’s prosocial behaviors \citep{ross_maternal_1990, smetana_toddlers_1989}, consistent with stereotypes that boys take risks and girls are nice to others

Could this be because of inborn sex differences in children’s behaviours? for example, parents are more likely to use some physical force to discipline boys than girls because boys are more physically active or challenging and therefore elicit such responses from adults \citep{verhoeven_parenting_2010} But sex differences in children’s behaviours are absent/small in infancy \citep{else-quest_gender_2006}, slowly emerge: evidence: actor babies were treated differently by adults (who had a baby of the same age) based on the pink or blue color of the infants’ clothes (i.e., perceived sex) rather than their actual sex, and parents are largely unaware of that happening \citep{culp1983comparison}

“Indirect gendered parenting practices convey gender-stereotyped messages to children about others or about general gendered expectations or opinions” 

Vicarious social learning theory: children pick up on gendered evaluative messages regarding others’ actions \citep{bandura_influence_1965}

Evidence: get parents to read books with their children that contain pictures designed to elicit gender-relevant talk, results show that even though parents rarely make explicit gendered comments to children, they send gendered messages more subtly, by differentially evaluating and labeling stereotypical and counterstereotypical behaviors \citep{deloache_three_1987, friedman_mothers_2007, gelman_mother-child_2004, endendijk_boys_2014, van_der_pol_fathers_2015}

modeling is another form of indirect gendered message: source of information about gender roles \citep{bussey_social_1999}. influences children’s notions of what is typically male or female even when not told explicitly, and children are motivated to imitate their parents, can play out behaviours that fit the overall picture of gendered behavioral patterns (guidelines for behaviour in similar situations) \citep{bussey_social_1999}

children from families with traditional gender roles have more gender-stereotypical expectations \citep{sinno_moms_2009}.

According to gender schema theory, parents’ gender stereotypes predict the extent to which they engage in gendered parenting, which in turn predicts children’s gender stereotypes and gendered behaviors \citep{bem_gender_1981}.

evidence: in early childhood, parents’ gender stereotypes may be associated with gendered parenting \citep{endendijk2016gender, friedman_mothers_2007, endendijk_boys_2014}

evidence: fathers with more stereotypical gender attitudes used more physical control (typically seen as appropriate for boys) with sons than with daughters, and this pattern predicted stereotypically greater aggression in sons than in daughters \citep{endendijk_gender_2017}

Research ideas:

\begin{itemize}
\item how are implicit and explicit intertwined?

\item What dimensions we measure matters: shapes the nature of scientific knowledge that emerges. If we only look at explicit dimensions, then we conclude there’s no differences

\item Understanding parents’ motives: do they gender because they explicitly think they should to prepare their child for the world (“socialized in a way that prepares them for adult gender roles and society’s gendered expectations.” In any case, parents “may not want to reveal this goal if they believe it contradicts society’s ideology of gender equality” Or they make differences because they’re victims of unconscious gender stereotypes pervasive in many societies \citep{miller_womens_2015}? but weak associations between implicit gender stereotypes in mothers and gendered talk in interactions with child \citep{endendijk_boys_2014}

\item what happens when societies periodically undergo various evolutions and revolutions regarding gender roles?

\item what happens in non-Western cultures or ethnic-minority families of non-Western backgrounds?

\item by parents’ socioeconomic status (people from lower income backgrounds more likely to endorse more traditional gender roles \citep{ex_maternal_1998}

\item the question of outcomes of gendered parenting: different ideologies about the costs and benefits of gendered parenting:

\item on the one hand, gendered parenting teaches children about the reality of gender role expectations, prepares them for socially adaptive functioning and may promote greater well-being

\item on the other hand, parenting based on stereotypes instead of actual abilities and interests, talent may be wasted and people may be forced into lifestyles and careers that deny personal identities

\end{itemize}


\section{Theory}

\subsection{Doing gender \citep{west1987doing}}

\subsubsection*{Nutshell}

There's sex, sex category, and gender. You do gender to fall in the sex category you want. And you do gender through everything you do, through every interaction (you're always a man/woman, you're not always a student/partner/etc), so as not to be discredited, to avoid threats to the security of your sex category. Doing gender also means creating differences between boys and girls? Why do we need two categories at all?

"Doing gender furnishes the interactional scaffolding of social structure, along with a built-in mechanism"

\subsubsection*{Check out}

\bibentry{cahill1986a}

\subsubsection*{Main ideas: to complete}

\subsection{Social–cognitive theory of gender development and differentiation \citep{bussey_social_1999}}

\subsubsection*{Nutshell}

In trying to explain why gender differences happen, different theories have focused on the importance of different elements, independent from one another, social–cognitive theory puts forward the relation and the intercausality between different systems, all forming and reinforcing gender differences: self, family, workplace, media, peers, etc: operate interdependently, not as disjoined entities

\subsubsection*{Main ideas}

This article addresses the psychosocial determinants and mechanisms by which society socializes male and female infants into masculine and feminine adults

Why gender matters: because some of the most important aspects of people's lives are heavily prescribed by societal gender-typing added importance because many of the attributes and roles selectively promoted in males and females tend to be differentially valued with those ascribed to males generally being regarded as more desirable, effectual, and of higher status \citep{Berscheid1993}

Social cognitive theory of gender role development and functioning integrates psychological and sociostructural determinants within a unified conceptual framework
\begin{itemize}
 \item multifaceted social transmission model
 \item takes a life-course perspective
 \end{itemize}
 
 Describes theories, and the fact that little evidence support them:
\begin{itemize}
 \item Psychoanalytic Theory: Identification with the same-sex parent is presumed to resolve the conflict children experience as a result of erotic attachment to the opposite-sex parent and jealousy toward the same-sex parent, etc \citep{chodorow1978}

 \item Cognitive-Developmental Theory: gender identity is postulated as the basic organizer and regulator of children's gender learning \citep{kohlberg1966} Cognitive consistency is gratifying, so individuals attempt to behave in ways that are consistent with their self-conception. gender constancy
 
 \item Gender Schema Theory: Once the schema is developed, children are expected to behave in ways consistent with traditional gender roles. The motivating force guiding children's gender-linked conduct, as in cognitivedevelopmental theory, relies on gender-label matching in which children want to be like others of their own sex \citep{bem1981, markus1982}. \cite{martin1981}'s gender identity
 
 \item Biological Theories: Evolutionary psychology, gender differentiation as ancestrally programmed \citep{archer1996, buss1995, simpson1997} women have come to invest more heavily than men in parenting roles \citep{trivers1972}. OR hormonal influences the organization of the neural substrates of the brain, including lateralization of brain function, but differences aren't that big \citep{bryden1988, halpern1992, kinsbourne1983}. Plus, high heritability does not mean unmodiflability by environmental means.
 
 \item Sociological Theories: gender is a social construction rather than a biological given. social and institutional practices. Gender stereotypes shape the perception, evaluation, and treatment of males and females in selectively gendered ways that beget the very patterns of behavior that confirm the initial stereotypes \citep{geis1993} Many sociologists reject the dichotomous view of gender, in that the similarities between men and women in how they think and behave far exceed the differences between them \citep{epstein1988, gerson1990, west1991} and vast differences among men depending on their socioeconomic class, education, ethnicity, and occupation. The exaggeration of the nature and extent of gender differences, the theorists argue, promotes the social ordering of gender relations and serves to justify gender inequality, occupational stratification and segregation. Not all people of the same socioeconomic status, and who live under the same opportunity structures, social controls, familial, educational and community resources, and normative climate, behave in the same way. The challenge is to explain adaptational diversity within sociostructural commonality. People are producers as well as products of social systems. Social structures are created by human activity \citep{bandura1997, bandura1999, giddens1984}. The structural practices, in turn, impose constraints and provide resources and opportunity structures for personal development and functioning.
 \end{itemize}
 
Social cognitive theory addresses itself to a number of distinctive human attributes:
\begin{itemize}
\item capability for symbolization

\item advanced capability for observational learning

\item self-regulatory capability
 
\item self-reflective capability
\end{itemize}

Social cognitive theory does not assume an equipotential mechanism of learning \citep{bandura1986}. In addition to biological biases, some things are more easily learnable

\cite{gould1987} noted, biology sets constraints that vary in nature, degree, and strength across different spheres of functioning; however, in most domains the biology of humans permits a broad range of cultural possibilities.

A biologically deterministic view has problems not only with cultural diversity, but with the rapid pace of social change.

Gender development is explained in terms of triadic reciprocal causation: the \textit{personal} contribution includes gender-linked conceptions, behavioral and judgmental standards, and self-regulatory influences; \textit{behavior} refers to activity patterns that tend to be linked to gender; and the \textit{environmental} factor refers to the broad network of social influences that are encountered in everyday life.

The relative contribution of each of the constituent influences depends on the activities, situations, and sociostructural constraints and opportunities: under social conditions in which social roles, lifestyle patterns, and opportunity structures are rigidly prescribed, personal factors have less leeway to operate.

Sociocognitive Modes of Influence

\begin{itemize}
\item modeling. A great deal of gender-linked information is exemplified by models in one's immediate environment such as parents and peers, and significant persons in social, educational, and occupational contexts.
Modeling is not simply a process of response mimicry, rule-governed action patterns differ in specific content and other details, but they embody the same underlying rule. Models exemplify activities considered appropriate for the two sexes. Children can learn gender stereotypes from observing the differential performances of male and female models. Observers pay greater attention to and learn more about modeled conduct that they regard as personally relevant \citep{kanfer1971}. Because adherence to gender roles is socially stressed more for boys than for girls across the life span, boys tend to pay more attention to samegender models than do girls \citep{slaby1975}.

\item enactive experience. It relies on discerning the gender linkage of conduct from the outcomes resulting from one's actions. Gender-linked behavior is heavily socially sanctioned in most societies. Therefore, evaluative social reactions are important sources of information for constructing gender conceptions.
 
\item direct tuition. It serves as a convenient way of informing people about different styles of conduct and their linkage to gender. Moreover, it is often used to generalize the informativeness of specific modeled exemplars and particular behavioral outcome experiences. direct tuition is most effective when it is based on shared values and receives widespread social support. Models, of course, do not often practice what they preach. The impact of tuition is weakened when what is being taught is contradicted by what is modeied \citep{hildebrandt1973, mcmanis1968, rosenhan1968}.
\end{itemize}
	
some modes of influence are more influential at certain periods of development than at others. Modeling is omnipresent from birth. Infants are highly attentive to modeling influences and can learn from them, especially in interactive contexts \citep{bandura1986, uzgiris1992}. As children gain mobility and competencies to act on the environment, they begin enacting behavior that is socially linked to gender and experiencing social reactions. They regulate their behavior accordingly. As they acquire linguistic skills, people begin to explain to children what is appropriate gendered conduct for them.

Learning conceptions through modeling is faster than from enactive experience \citep{bandura1986, debowski1999} In the enactive mode, conceptions of gendered conduct must be constructed gradually by observing the differential outcomes of one's actions

Tuition is weakened by the abstractness and the complexity of language, especially for young children. Verbal instruction alone, therefore, has less impact on conception acquisition than does modeling \citep{rosenthal1978}

Children have to gain predictive knowledge about the likely social outcomes of gender-linked conduct in different settings, toward different individuals and for different pursuits.

simply knowing the stereotypes, which increase with age, does not necessarily motivate children to act in accordance with them. Indeed, a meta-analytic study showed that as children become increasingly knowledgeable about gender role stereotypes, they believe less strongly, especially girls, that those stereotypes should exist \citep{signorella1993}.

After self-regulatory functions are developed, children guide their conduct by sanctions they apply to themselves. They do things that give them self-satisfaction and a sense of self-worth. They refrain from behaving in ways that violate their standards to avoid self-censure. The standards provide the guidance; the anticipatory self-sanctions provide the motivators. Self-sanctions thus keep conduct in line with personal standards.

As children become aware of the social significance attached to gender, they increasingly attend to this aspect of their behavior \citep{serbin1986}. In mixed-sex groups, children are more likely to monitor behavior according to its gender linkage. Compared with girls, boys monitor their behavior on the gender dimension more closely because, as already noted, they are more likely to be reproached for conduct that deviates from their gender (Martin, 1993). Moreover, boys have a strong incentive to oversee male-linked behavior because it usually carries higher status and power than female-linked behavior \citep{fagot1985, fagot1993}.

Among the mechanisms of agency, none is more central or pervasive than people's \textbf{beliefs in their capabilities} to produce given levels of attainments. Unless people believe they can produce desired effects by their actions, they have little incentive to act or to persevere in the face of difficulties. Perceived efficacy is, therefore, the foundation of human agency.

People's beliefs in their efficacy can be developed in four major ways.

\begin{itemize}
\item graded mastery experiences. Successes build a robust belief in one's personal efficacy. Failures undermine it.

\item  social modeling. Models transmit knowledge, skills, and strategies for managing environmental demands. Seeing people similar to oneself succeed by sustained effort raises observer's beliefs in his or her own capabilities.

\item Social persuasion: Expressing faith in people's capabilities raises their beliefs that they have what it takes to succeed; however, effective efficacy builders do more than convey positive appraisals. They structure activities in ways that bring success and do not place people prematurely in situations likely to bring failure.

\item reduce people's stress and depression, build their physical strength, and change misinterpretations of their physical states.
\end{itemize}
	
The effects of goals, outcome expectations, causal attributions, and perceived environmental opportunities and impediments on motivation are partly governed by beliefs of personal efficacy \citep{bandura1991a, bandura1997}.

The power of efficacy beliefs to affect the life paths of men and women through selection processes is most clearly revealed in studies of career choice and development \citep{bandura1997, hackett1995}. Those who have a strong sense of personal efficacy consider a wide range of career options, show greater interest in them, prepare themselves better for different careers, and have greater staying power in their chosen pursuits \citep{lent1994}.

The pervasive stereotypic practices of the various societal subsystems, which we examined earlier, eventually leave their mark on women's beliefs about their occupational efficacy.

Women's beliefs about their capabilities and their career aspirations are shaped by undermining social practices within the family, the educational system, peer relationships, the mass media, the occupational system, and the culture at large \citep{bandura1997, betz1987, dweck1978, eccles1989, gettys1981, hackett1981, jacobs1989, mcghee1980, phillips1990, signorielli1990}.

Simply invoking the gender stereotype can undermine women's efficacy to make good use of the mathematical competencies they possess \citep{steele1997}.

computers have become masculinized. As a result, boys receive encouragement from an early age by parents and teachers to develop computer literacy. As a consequence, they regard computer skills as more important to their career development than do girls \citep{hess1985, lockheed1985, ware1985} men are benefiting much more than women from these technological advancements \citep{gallie1991}.

It should also be noted that the variability within sexes exceeds the differences between them.

although gender identity and constancy were posited as the factors governing gender development, the mechanisms by which they come into being remain unspecified. They are simply assumed to be products of interaction with the environment

None of gender identity, gender constancy, nor gender stereotypic knowledge predicts gender-linked conduct.

Even during the early years, fathers are more stereotypic socializers than are mothers \citep{langlois1980, snow1983}.

For boys, there is little conflict between their own valuation of their gender and societal valuation of it. For girls, however, although they may value being a girl and gender-linked activities, they very early recognize the differential societal valuation of male and female roles \citep{kuhn1978, meyer1980}. Consequently, women have some incentive to attempt to raise their status by mastering activities and interests traditionally typed as masculine. Even at the preschool level, girls show greater modeling after the other gender than do boys.

Gender role learning requires broadening gender conceptions to include not only appearances but clusters of behavioral attributes and interests that form lifestyle patterns and social and occupational roles as well. Knowledge about gender roles involves a higher level of organization and abstraction than simply categorization of persons, objects, and activities in terms of gender. To complicate matters further, the stylistic and role behaviors that traditionally typify male and female orientations are not uniformly gender linked. Many men are mild mannered and some females are aggressive. As a result, children have to rely on the relative prevalence of exemplars and the extent to which given activities covary with gender.

In some of the current theorizing, the peer group is singled out as the prime socializing agency of gender development \citep{leaper1994, maccoby1990, maccoby1998}. The view of the peer group as the ruling force is coupled with the disputable claim that parents do not differ in their gendered practices with sons and daughters \citep{lytton_parents_1991}. The peer group is not an autonomous agency untouched by familial and other social influences.

Parents encourage peer associations that uphold parental standards and support valued styles of behavior in contexts in which the parents are not present \citep{bandura1959, elkin1955}.

Another explanatory possibility is that boys and girls are, for some reason, attracted to different types of toys and activities: question of the source of the attraction

The school functions as another primary setting for developing gender orientations. With regard to shaping gendered attributes, teachers criticize children for engaging in play activities considered inappropriate for their gender \citep{fagot1977}. As in the case of parents and peers, teachers foster, through their social sanctions, sharper gender differentiations for boys than for girls. Teachers also pay more attention to boys than girls and interact with them more extensively \citep{ebbeck1984, morse1985}. From nursery school through to the early elementary school years, boys receive more praise as well as criticism from teachers than girls \citep{cherry1975, simpson1983}. The nature of the social sanctions also differ across gender. Boys are more likely to be praised for academic success and criticized for misbehavior, whereas girls tend to be praised for tidiness and compliance and criticized for academic failure. This differential pattern of social sanctions, which can enhance the perceived self-efficacy of boys but undermine that of girls, continues throughout the school years \citep{eccles1987}.

Changing gender roles pose challenges on how to strike a balance between family and job demands for women who enter the workforce. The effects of juggling dual roles are typically framed negatively on how competing interrole demands breed distress and discordance. Much has been written on the negative spillover that women's job pressures have on family life but little on how job satisfaction may enhance family life. Research by \cite{ozer1995} speaks to this issue. Married women who pursued professional, managerial, and technical occupations were tested before the birth of their first child for their perceived self-efficacy to manage the demands of their family and occupational life. Their physical and psychological well-being and the strain they experienced over their dual roles were measured after they had returned to work. Neither family income, occupational workload, nor division of child-care responsibility directly affected women's well-being or emotional strain over dual roles. These factors were contributors, but they operated through their effects on perceived self-efficacy. Women who had a strong sense of coping efficacy (i.e., that they can manage the multiple demands of family and work, exert some influence over their work schedules, and get their husbands to help with various aspects of child care) experienced a low level of physical and emotional strain, good health, and a more positive sense of well-being. Neither conceptual schemes nor empirical studies have given much attention to the positive spillover effects of women's satisfying work lives on their home lives.
 
The interplay of personal and sociostructural impediments creates disparity in the distribution of women and men across occupations that differ in prestige, status, and monetary return. All too often, this leads to devaluation not only of women's work but the "feminized" occupations as well \citep{reskin1991}.

glaring absence of research on how fathers juggle the dual demands of the workplaces and housework and child care. When men and women do share family responsibilities, the criticism they receive discourages nontraditional family life \citep{deutsch1998}.


\section{Context}

 \subsection{The Two-Part Gender Revolution, Women’s Second Shift and Changing Cohort Fertility  \citep{frejkaTwoPartGenderRevolution2018}}
 
 \subsubsection*{Nutshell}

Paper a bit too focused on fertility, but interesting summary of gender revolution in two parts and of differences in attitudes versus behaviour.

\subsubsection*{Main ideas}

Part 1: growth of female labor force participation
Part 2: growth in men’s involvement in the tasks at home
\citep{goldscheider2015gender}

women’s “second shift” = the gap between the two stages \citep{hochschild1989}

Prior to the middle of the 20th century, few married women worked outside the home, at least in Sweden and among the white majority in the United States, two countries recognized as leaders in the growth of female labor force participation \citep{stanfors2017}. The late nineteenth century and the first half of the twentieth century were the period of the “separate spheres,” when married women remained in the home if at all possible, even as the growth in industrial and commercial jobs drew men away from the subsistence household economy. During this period women pursued many productive roles, including replacing their husbands’ chores (e.g., drawing water, maintaining kitchen gardens), in addition to their own (e.g., spinning, sewing, washing), together with increased responsibilities for the daily activities of children. These roles evolved over time, and generally became less onerous, with the growth of cities, together with the spread of running water and electrification and children’s education \citep{cowan1983}. Nevertheless, preconditions for eventual employment roles were developing, as women’s education increased, as well as their employment experience prior to marriage \citep{stanfors2017, goldin1990}.  The growth of labor force participation by women during the second half of the 20th century, especially by married women and even married mothers, as a result, was rapid. This trend has been frequently documented \citep{pottbuter1993, rosenfeld1996, spain1996}. In the early years of the increase in female labor force participation, a research consensus emerged that it was the result of the growth in demand for “female” workers \citep{oppenheimer1970, goldin1990}. More recent studies have highlighted the positive effects of subsidized childcare 
\citep{pettit2005}.

The growth in men’s involvement in the home is much less well documented than the growth of female labor force participation. This is in part because information on domestic activities has been much less regularly collected than labor force information.

 \subsection{Equal sharing or not at all caring? Ideals about fathers’ family involvement and the prevalence of the second half of the gender revolution in 27 societies  \citep{edlund2023equal}}

\subsubsection*{Nutshell}

With two survey questions, they show how people think parent should split work at home and at work. They create 6 clusters (from more traditional breadwinners to dual earner dual carer) and show which are more popular in different countries. Shows that some setups are considered more progressive in some countries, while considered more conservative in others. This papers talks to the Second Gender Revolution: are people tending towards the idea of men being at home? The main conclusion is that this part of the revolution is not widespread.

\begin{figure}[H]
	\centering
	\noindent\includegraphics[width=\textwidth]{/Users/alexsheridan/Documents/Work/PhD/latex/figures/edlung2023tab1.png}
\end{figure}

In the UK, a preference for variants of the male breadwinner/female homemaker family model is also widespread. This model constitutes the conservative alternative in this group of countries. However, we also find support for the one and-a-half-earner family model, which represents the major progressive alternative.

In France, The progressive alternative is a mixture between the one-and-a-half-earner family model, the full-time family model and the dual-earner/dual-carer family model, while the conservative alternative is made up of clusters 2 (ideological male breadwinner) and 6 (traditional male breadwinner). Cluster 1 (one-and-a-half-earner family) constitutes a middle alternative. There are clear educational differences in support of both clusters 2 and 6, with individuals with lower education being more in favour and those with higher education less in favour of the male breadwinner/female homemaker family models. The opposite is true for cluster 4: individuals with higher education are more supportive of this cluster than those with low education. There are also clear age differences in the support for clusters 4 and 6 that point in the same direction, with older people being more in favour of the conservative alternative.

\begin{itemize}
	\item Method: latent class analysis
	\item Data: International Social Survey Programme (ISSP) of 2012
	\item Region: Europe + US + Australia + Canada
\end{itemize}

\begin{figure}[H]
	\centering
	\noindent\includegraphics[width=\textwidth]{/Users/alexsheridan/Documents/Work/PhD/latex/figures/edlung2023tab2.png}
\end{figure}

\begin{figure}[H]
	\centering
	\noindent\includegraphics[width=\textwidth]{/Users/alexsheridan/Documents/Work/PhD/latex/figures/edlung2023surveyqs.png}
\end{figure}

\subsubsection*{Check out}

\bibentry{hook2020}

\medskip

\noindent \bibentry{friedman2015}

\medskip

\noindent \bibentry{europeancommission2019}

\medskip

\noindent \bibentry{pailhe2021}

\subsubsection*{Main ideas}

The aim is to delineate the extent to which different countries are heading towards a gender-equal and socially sustainable society – in other words, a realization of the second half of the gender revolution

Research on the second half of the gender revolution, which has grown over time, uses empirical data on behaviour and/or attitudes and beliefs. Nevertheless, especially when it comes to data on attitudes or beliefs about family and care, systematic large-n crossnational research on men’s family involvement is rare. One important reason for this lack is that truly cross-national comparative data treating men and women on equal terms have been scarce.

We've focused a lot on women's gainful employment, and on distribution of housework

Although the association between attitudes and behaviour may be less than straightforward, few would agree that they are unrelated. Under certain conditions, the association can even be strong \citep{ajzen2005}.

\cite{aassve2014}, using European cross-national data, observe that attitudinal dispositions about gender relations are strong predictors of actual gender-related behaviour.

the ability of social policy to structure and convey normative messages about gender specialization versus dual/overlapping social roles \citep{bergqvist2017, daly2003, lohmann2016}. As systems of rules and regulations, institutions embody national traditions and previous power struggles between social actors.

redistribution of resources and as a carrier of norms \citep{korpi1998, mettler2004, svallfors2007}. Consequently, through their institutional frameworks, countries facilitate or even promote different ‘family models’.

while national institutions affect public opinion and the interests of specific social groups, it is equally true that public opinion and group interests affect – often, but not exclusively, via general elections – government behaviour and thereby the structure of national institutions \citep{brooks2007, mettler2004, rothstein1998}.

public opinion: a potential force for societal change or the preservation of the current circumstances.

latent class analysis (LCA), which is a suitable method for identifying qualitatively different configurations of categorical variable responses \citep{magidson2001}; that is, it can reveal the patterns of ideals and attitudes that are dominant in different countries

the quantitative strategy studies the relationships between the ideal typical family preferences retrieved from the LCA and two well-recognized key indicators that are used in international indices of female empowerment – labour force participation and political representation in parliament \citep{klasen2011, phillips1998}

modernization theory, we also include a measure of the countries’ wealth (BNI/capita).

Cluster 1, one-and-a-half-earner family model, represents around 25 per cent

three clusters of preferences in which the mother is mainly understood to be a housewife, these three clusters constitute almost 50 per cent of the respondents:
\begin{itemize}
	\item ideological male breadwinner/female homemaker family model (Cluster 2)
	\item child-centred male breadwinner/female homemaker family model (Cluster 3)
	\item traditional male breadwinner/female homemaker family model (Cluster 6)
\end{itemize}

Cluster 4: shared responsibility for both paid work and care: the dual-earner/dual-carer family model, represents around 14 percent

Cluster 5: contains individuals with a strong work orientation: the full-time family model, represents 13 per cent

Construction of a conservative-progressive scale (CPS-1), based on the support for clusters 1, 4 and 5
CPS-1 = (C1*0.50) + C4 + C5

In conclusion, the higher the gender equality in politics and employment and the higher the degree of modernization, the lower the proportion of citizens in a country that prefer a traditional gendered organization of a family’s work and care obligations.

Make 6 groups of countries
\begin{itemize}
	\item Bulgaria, Slovakia, the Czech Republic, Lithuania and Hungary where the level of support for the different kinds of male breadwinner/female homemaker family model (clusters 2, 3 and 6) is overwhelming, no progressive option stands out
	\item Austria, Latvia, the UK, the USA, Portugal, Australia and Ireland: preference for variants of the male breadwinner/female homemaker family model is also widespread, but also support for the oneand-a-half-earner family model, which represents the major progressive alternative
	\item Croatia, Poland, Canada and Slovenia is similar to the previous group, but the progressive alternative in this group is shared between the one-and-a-half-earner family model and the full-time family model.
	\item Germany-west, Switzerland and the Netherlands: a large part of the population supports the one-and-a-half-earner family model and the full-time family model, though some strive towards the more conservative male breadwinner/female homemaker model, while others want to move towards the dual-earner/dual-carer alternative
	\item Belgium, France and Finland: male breadwinner/female homemaker family model is less popular than in the previous groups, it's the conservative alternative. The progressive alternative is a mixture between the one-and-a-half-earner family model, the full-time family model and the dual-earner/dual-carer family model.
	\item Germany-east, Denmark, Norway, Iceland and Sweden: the male breadwinner/female homemaker family model is discarded. Instead, the one-and-a-half-earner family model stands out as the conservative alternative. The progressive alternative is either the full-time model or the dual-earner/dual-carer family model.
\end{itemize}

the higher the proportion of citizens holding progressive preferences, the more pronounced the differences in preferences between social categories.

In a large majority of the studied societies, citizens believe that women should be responsible for the home and family, while men’s family involvement is seen as less important.

However, it is only in a group of six Eastern European societies that the male breadwinner/female homemaker ideal stands unchallenged. Across the other societies, we find three different alternatives that represent progressive alternatives to this ideal: women’s part-time work; a full-time ideal for both women and men; and a dual-earner/dual-carer family ideal, with shared responsibility for paid work and care (shared part-time) for both mothers and fathers.

The most prominent institutional changes over the last decades have occurred in the more progressive countries. Apart from affecting behaviour in these citizenries, there are indications that these developments have been accompanied by salient family-related political rhetoric and public debate. Therefore, in progressive countries, the public is likely to be comparatively more informed

\section{Empirical findings}

\subsection{A comparison of observed and reported adult-infant interactions: Effects of perceived sex \citep{culp1983comparison}}

\subsubsection*{Nutshell}

Parents treat the child actor differently when they're dressed as a boy or a girl: direction of gaze, facial expression, physical contact with the infant, and toy used. Mothers especially unaware they're doing it. Shows subtle sex-typing of infants by adults.

\begin{itemize}
	\item Method: experiment and interviews, with an infant actor around 6 months old
	\item Data: 16 parents (8 couples) whose youngest was less than 2.5 yo
	\item Region: US?
	\item Who are these parents? Education? Very few details
\end{itemize}

\subsubsection*{Main ideas}

Not much, 4 page paper

\subsection{Parenting and children's externalizing behavior: Bidirectionality during toddlerhood \citep{verhoeven_parenting_2010}}

\subsubsection*{Nutshell}

No support for bidirectionality when looking at boys's externalising: 
Boys' externalizing behavior predicted parent-reported support, but parenting didn't predict externalising.

\begin{itemize}
\item Method: SEM, questionnaires at 17, 23 and 29 and 35 months of age, measuring 11 scales from existing valid and reliable instruments that represent five parenting dimensions (support, lack of structure, positive discipline, psychological control, and physical punishment), and Child Behavior Checklist 1 1⁄2–5 for the externalising behaviours
\item Data: 104 intact families with a toddler son, mostly white and educated
\item Region: Netherlands
\end{itemize}
 
\subsubsection*{Main ideas}

A long history of research on parent–child relationships has been based on the assumption that parents influence their children to a greater extent than children influence their parents \citep{pettit1997}; parents were conceptualized as the primary socializing agents of their children, and children were regarded as the passive recipients of this socialization \citep{perlman1997, pettit1997}.

Bell's (1968) seminal article on child – rather than a parent-effects model (i.e., the idea that children elicit certain types of parenting from their mothers and fathers).

both the transactional effects model \citep{sameroff1975} and the control system model \citep{bell1977, bell1986} posit recurrent bidirectional influences between the parent and the child. According to these models, child behavior evokes certain parental reactions which, in turn, influence future child behavior. In this way, child and parental behaviors enter into a system of bidirectional influences \citep{lytton1990}.

According to \cite{scarr1992}, as long as parents are ‘good enough’, it does not matter in which family children grow up, as parents have few differential effects on children.

the significance of parenting behavior for children's externalizing behaviors may not be evident before children enter school \citep{scaramella2004}. The developmental importance of the early parent–child relationship is that children learn strategies for interacting with others (i.e., other children, teachers), which affects future behavior and relationships. Thus, it might be that the effects of parenting on children's externalizing behaviors are not yet visible at this early age.

As suggested by \cite{sroufe2000} and \cite{woodworth1996}, the myriad of developmental changes that take place during the child's second and third year seem likely to draw fathers more actively into parenting.


\subsection{Gender Stereotypes in the Family Context: Mothers, Fathers, and Siblings \citep{endendijk2013}}

\subsubsection*{Nutshell}

Looks at implicit and explicit gender stereotypes of parents and children. Mothers had stronger implicit gender stereotypes, while fathers had more explicit stereotypes: maybe because of  desirability bias? Lower maternal educational level was related to stronger explicit attitudes about gender in both parents. When mothers showed stronger gender stereotypes, their daughters also showed stronger gender stereotypes.

\begin{itemize}
\item Method: family systems model, get parents and children to take Action Inference Paradigm (AIP; both child and parents) and the Implicit Association Test (parent only), and parents: Child Rearing Sex-Role Attitude Scale
\item N: 172 families, parents with two children younger than 2.5 and 3.5 years old. Exclusion criteria were singleparenthood, etc.
\item Region: Netherlands
\item Slight bias in the sample: fewer less educated families
\end{itemize}

\begin{figure}[H]
	\centering
	\noindent\includegraphics[width=\textwidth]{/Users/alexsheridan/Documents/Work/PhD/latex/figures/endendijk2013fig.png}
\end{figure}

\subsubsection*{Check out}

\bibentry{rothbaum1994}

\medskip

\noindent \bibentry{shaw1998}

\medskip

\noindent \bibentry{obrien2000}

\medskip

\noindent \bibentry{signorella1993}


\subsubsection*{Main ideas}

Gender stereotypes are widely held beliefs about the characteristics, behaviors, and roles of men and women \citep{weinraub1984development}. In the preschool period family  context and family experiences are important for gender stereotype development \citep{mchale_family_2003, witt1997}.

Evidence for explicit stereotypes \citep{mchale1999, turner1995}, but not so much implicit (unconscious)

not entirely clear whether implicit tasks indeed measure a person’s own stereotypes, or culturally shared attitudes \citep{dehouwer2009}, but for controversial subjects like gender and race, U.S. studies with adults have shown that implicit stereotypes are better predictors of behavior than explicit self-reported stereotypes  \citep{nosek2002a, nosek2002b, nosek2005, rudman2004}, because explicit reports may be biased by social desirability and a lack of awareness of own stereotypes  \citep{kunda2003, white2006}. Social desirability tendencies appear to be strongest among people with higher levels of education, because of their greater awareness of what are appropriate responses, according to a U.S. study with adults  \citep{krysan1998}. So, educational level of participants has to be taken into account when examining gender stereotypes.

Several U.S. studies found that by the age of 4 years stereotypes are well developed  \citep{fagot1992}, but it takes until about 8 years of age for gender stereotypes to become more complex, flexible and similar to adult stereotypes  \citep{martin1990, trautner2005}.

family systems perspective: each family member is influenced by the other family members  \citep{bowen1978}.

On siblings:
\begin{itemize}
	
	\item evidence from U.S. studies with preschool children that siblings have a profound effect on gender role socialization and explicit gender stereotypes  \citep{mchale_family_2003, rust2000, stoneman1986}.
	
	\item modeling or reinforcement of opposite gender attributes in mixed-gender siblings  \citep{rust2000, stoneman1986}. However, another U.S. study proposed that mixedgender siblings might have the strongest explicit gender stereotypes, because parents of mixed-gender children have the opportunity for gender-differentiated parenting  \citep{mchale1999}.
	
	\item younger siblings may exert their influence on the gender stereotypes of older siblings in the more passive way proposed in the study of \cite{mchale1999}, because infants are unlikely to be active reinforcers of gender attributes
	
	\item the opportunities for gendered comparisons of parents in mixed-gender families may also increase the likelihood of stronger parental attitudes about gender.
	
	\item having both a boy and a girl may make the wish to treat the two genders equally and the desire for happy and successful futures for both of their children more important for fathers, resulting in more egalitarian attitudes.
\end{itemize}

considerable evidence, mostly from U.S. studies, that parents treat boys and girls differently \citep{chaplin2005, lytton_parents_1991, martin2005}.

Gender schema theory \citep{bem1981, bem1983} suggests that the way parents behave towards their children is guided by gender schemas that consist of gender-typed experiences.

if the gender schemas of parents consist of stereotypical associations they are more likely to show gender-differentiated parenting. Gender schema theory proposes that children will internalize these gender-typed experiences in a gender schema of their own \citep{gelman_mother-child_2004, witt1997}.

A meta-analysis with samples from various countries found a small influence of parental gender schemas on their child’s attitudes about gender \citep{tenenbaum2002}. Most of the studies in this meta-analysis used explicit measures

wo U.S. studies point to a more prominent role for implicit attitudes about gender, because parents are largely unaware of their different behaviors to boys and girls \citep{culp1983comparison} and many parents reject common gender stereotypes, but still apply these stereotypes implicitly as reflected by their approval or disapproval of children’s toy preferences  \citep{freeman2007}.

preschool boys and girls may vary in their susceptibility to the rearing environment, according to a meta-analysis  \citep{rothbaum1994} and a study from the U.S.  \citep{shaw1998}.

mothers show different interactive behaviours with sons than with daughters  \citep{maccoby1990}.Mothers not only talk more to girls than to boys in general, as found in a U.S. study  \citep{leaper1998},but they also talk more about interests and attitudes to girls than to boys  \citep{boyd1989, noller1990}. mothers tend to be more engaged in play with their 6-, 9-, and 14-month-old daughters, whereas they spend more time watching boys and not interacting, as found in a U.S. study  \citep{clearfield2006}.

simply have more time to create gender-related experiences for children according to their own stereotypes  \citep{tenenbaum2002}

explicit stereotype measures are prone to social desirability  \citep{white2006} and women generally score higher on social desirability than men, according to a U.S. study \citep{hebert1995}: cultural gender roles influence the channels that are acceptable for stereotype expression, as found in a Swedish study  \citep{ekehammar2003}, rendering it less acceptable for women than for men to express explicit gender stereotypes. Women may have implicit gender stereotypes that are not considered appropriate to present explicitly, whereas men may use both their implicit and explicit channel in parallel.

Boys and girls, however, did not differ from each other in the strength of their implicit gender stereotypes. Although this was not expected, this is in line with several U.S. studies that focused on explicit gender stereotype development in preschool children \citep{obrien2000, signorella1993}. Apparently, gender differences in attitudes about gender start to develop later in childhood, probably during the school years where peer influence becomes more pronounced and children encounter more gender-related experiences outside the home.

\subsection{Social class, gender, and contemporary parenting standards in the United States: Evidence from a national survey experiment \citep{ishizuka2019social}}

\subsubsection*{Nutshell}

Parents of different social classes all have preference for concerted cultivation, even though they behave differently. Maybe low class parents have less resources to do concerted cultivation and settle for accomplishment of natural growth? All parents also expect the same intensiveness from mothers and fathers.

Nationally representative data describing contemporary cultural norms relating to mothers’ and fathers’ parenting, and how they vary by social class

\begin{itemize}
\item Method: experiment using a survey
\item N: 3,600 parents, nationally representative
\item Region: US
\end{itemize}

\subsubsection*{Check out}


\bibentry{mclanahan2004}

\medskip

\noindent \bibentry{heckman2006}

\medskip

\noindent \bibentry{weininger2015}

\medskip

\noindent \bibentry{calarco2014}

\medskip

\noindent \bibentry{musick2016}

\medskip

\noindent \bibentry{bianchi2006}

\medskip

\noindent \bibentry{ramey2010}

\medskip

\noindent \bibentry{nelson2010}

\newpage
\begin{figure}[H]
	\centering
	\noindent\includegraphics[width=\textwidth]{/Users/alexsheridan/Documents/Work/PhD/latex/figures/ishizuka2019fig1.png}
\end{figure}

\newpage
\begin{figure}[H]
	\centering
	\noindent\includegraphics[width=\textwidth]{/Users/alexsheridan/Documents/Work/PhD/latex/figures/ishizuka2019fig2.png}
\end{figure}

\subsubsection*{Main ideas}

parenting behaviors remain deeply divided by social class and gender: spend more time with them \citep{sayer2004parents, england2013}, but they also spend that time differently \citep{kalil2012}.

In theory could play a critical role in reproducing inequalities across generations \citep{heckman2006, mclanahan2004} because parenting behaviors predict a wide range of child outcomes.

And women still do a lot more, which keeps contributing to gender inequalities in the labour market \citep{correll2007motherhoodpenalty, england2005}.

At the same time, there's a cultural shift toward norms of timeintensive, child-centered parenting, particularly for mothers and among middleclass parents \citep{alwin1989changes, bianchi2006, lareau2003unequal, hays1996, nelson2010, ramey2010, sayer2004parents, senior2014, warner2005}.

Some scholars argue that parents differ by social class in how they conceive of good parenting \citep{england2013, calarco2014, lareau2003unequal, weininger2015}, whereas others contend that parents of different social classes are now similarly supportive of intensive parenting \citep{bennett2012, chin2004, edin2013, hays1996, waller2010}.

separate spheres ideology of intensive mothering and male breadwinning persists? \citep{hays1996, townsend2002}, yet others highlight the emergence of gender-egalitarian ideals of shared parenting and involved fatherhood \citep{edin2013, gerson2010, pedulla2015, waller2010}.

Need to study social class differences in cultural conceptions of good parenting are theorized to contribute in part to differences in parenting behavior \citep{calarco2014, lareau2003unequal, weininger2015, england2013}, and cultural norms may either promote or discourage genderegalitarian behavior change \citep{goldscheider2015gender, cherlin2016, espingandersen2015}

influential ethnographic study, for example, \citep{lareau2003unequal} argues that parents of different social classes have distinct “cultural logics of childrearing.”

In another ethnographic study, \cite{calarco2014} makes a related argument that middle-class and working-class parents have different logics of action—understandings of the kinds of skills and orientations that are appropriate—in schooling contexts, and so different strategies of action when interacting with teachers.

Ethnographies vs quantitative

\begin{itemize}
\item Ethnographies: provide direct evidence of differences, but small, non-random samples

\item Quantitative: indirect evidence: researchers argue that education differences in parenting behaviors that persist after statistically controlling for measures of time, money, and institutional constraints—such as income, work schedules, and neighborhood characteristicscan be interpreted as evidence of different “cultural orientations” or “cultural conceptions of appropriate parenting” \citep{weininger2015, england2013}.
\end{itemize}

The studies that say that parents want to do the same but can't: 

\begin{itemize}

\item other studies contend that parents of different social classes conceive of good parenting similarly. \cite{hays1996}

\item Recent qualitative studies also provide evidence that lowincome fathers have embraced middle-class ideals that emphasize love, communication, emotional involvement, and quality time \citep{edin2013, waller2010}.

\item  \cite{chin2004} find that parents of different social classes have unequal resources but similar desires to cultivate children’s talents

\item  \cite{bennett2012} argue that differences in financial resources and community organizations—not cultural conceptions of good parenting—are the primary contributors to class differences in children’s organized activity participation
\end{itemize}

Because gender stereotypes arise in part from observations of what men and women typically do \citep{ridgeway2011}, gender inequalities in parenting responsibilities are theorized to anchor the meaning of “good” mothering to more involvement than “good” fathering.

For example, a mother who contacts a teacher to advocate for her child may only be considered a “good” mother (because her behavior is compared with that of other mothers, who are typically more involved), but the same behavior may be judged as “excellent” when done by a father (because his behavior is compared with that of other fathers) \citep{kobrynowicz1997}

qualitative evidence suggests that mothers are held to higher standards than fathers, for example \cite{deutsch1998}.

emergence of gender-egalitarian parenting norms: women’s “second shift” has become harder to justify and pressure for men to share at home has grown \citep{espingandersen2015, deutsch1999, gerson1993}.

Changes in the importance of fatherhood in men’s lives

Recent time diary data show that fathers experience more happiness, less sadness, less fatigue, and more meaning in time spent with children than without \citep{musick2016}.

Although some researchers posit that gendered preferences may explain differences by child gender in father involvement or parental investments \citep{lundberg2007child, bertrand2013}, differences in parenting by child gender appear to be relatively small, and these measures do not align closely with key distinctions between concerted cultivation and natural growth parenting styles. Moreover, most research on social class and parenting does not report differences in parenting behaviors based on child gender \citep{calarco2014, weininger2015, chin2004, bennett2012}.

the spread of gender-egalitarian cultural norms may be an important—though not sufficient—condition for promoting gender-egalitarian behavior change in family life \citep{goldscheider2015gender, cherlin2016, espingandersen2015}.

Perhaps poor and working-class parents aspire to engage in concerted cultivation behaviors, but in a sense “settle” for natural growth, a parenting approach they generally regard positively, but one that is more compatible with their more time- and money-constrained circumstances.
raises the question of whether changing parents’ resources could bring about changes in their parenting behaviors.

different evaluation standards for mothers and fathers—and class differences in parenting attitudes—may manifest in evaluations of the frequency of parenting behaviors

Although the findings presented here indicate high contemporary cultural expectations for both mothers’ and fathers’ parenting, male breadwinning norms remain powerful and slow to change \citep{bertrand2015, killewald2016}.

Changes in the meaning of childhood, and mothers’ and fathers’ roles in children’s lives, tend to occur together with broader social, demographic, and economic transformations \citep{bianchi2006, zelizer1981}: 
\begin{itemize}
\item period changes such as increasing competition in college admissions \citep{ramey2010}, the growing economic value of college education \citep{bianchi2006}, and economic uncertainty \citep{lareau2003unequal, nelson2010}
\item cultural changes may also be cohort-driven, as suggested by increases across cohorts in gender-egalitarian attitudes about work and family roles and a declining emphasis placed on children’s obedience—particularly in cohorts born prior to 1950 \citep{cotter2011end, alwin1989changes}.
\end{itemize}

\subsection{Maternal Influences on Daughters' Gender Role Attitudes \citep{jan1998maternal}}

\subsubsection*{Nutshell}

A mother’s child-rearing style, as well as her own gender role attitudes, do influence the gender role attitudes a daughter develops. Level of education and mothers’ employment have indirect effects.

The main aim of this study is to analyze why mothers’ employment status and level of education are related to their daughters’ gender role attitudes. We hypothesized that these relations are mediated by a mother’s own gender role attitudes and her parenting style.

Tested whether (a) the mother’s own gender role attitudes and her child-rearing style were dependent on her employment status and level of education, (b) the mothers’ own gender role attitudes influenced their child-rearing style, and (c) nontraditional gender role attitudes of daughters were dependent on their mothers’ gender role attitudes and child-rearing style.

\begin{itemize}
	\item Method: mothers and daughters took questionnaires about attitudes and parenting style, then LISREL analysis, constructed latent variables
	\item N: 165 adolescents and their mothers, not representative, mostly white and two parents
	\item Region: Netherlands
\end{itemize}

\begin{figure}[H]
	\centering
	\noindent\includegraphics[width=\textwidth]{/Users/alexsheridan/Documents/Work/PhD/latex/figures/exjanssens1998fig.png}
\end{figure}

Why mothers and daughters: involved with one another on many fronts by virtue of their gender \citep{curtis1991}.

\subsection{“No Way My Boys Are Going to Be Like That!”: Parents’ Responses to Children’s Gender Nonconformity \citep{kane2006no}}

\subsubsection*{Nutshell}

Parents welcome what they perceive as gender nonconformity among their young daughters, while their responses in relation to sons are more complex. Heterosexual fathers are especially likely to be motivated in that accomplishment work by their own personal endorsement of hegemonic masculinity, while heterosexual mothers and gay parents are more likely to be motivated by accountability to others in relation to those ideals.

\begin{itemize}
	\item Method: qualitative interviews, 1999-2002
	\item N: 42 interviewees include 24 mothers and 18 fathers, with diverse family types, class, race, sexual orientations, education, with on average 2.5 children, focal child between the ages of three and five
	\item Region: US
\end{itemize}

\subsubsection*{Check out}

\bibentry{connell1987}

\medskip

\noindent \bibentry{kimmel1994}

\subsubsection*{Main ideas}

Children themselves become active participants in this gendering process by the time they are conscious of the social relevance of gender, typically before the age of two.

“to ‘do’gender is not always to live up to normative conceptions of femininity or masculinity; it is to engage in behavior at the risk of gender assessment.” I argue that many parents make efforts to stray from and thus expand normative conceptions of gender. But for their sons in particular, they balance this effort with conscious attention to producing a masculinity approximating hegemonic ideals: balancing act

Three bodies of literature provide foundations for this argument
\begin{itemize}
\item work documenting parental behaviors in relation to gendering children: predominantly quantitative, does less to explore the nuances of how parents make meaning around gender, to reveal what motivates parents as they participate in the social construction of their children’s gender, or to illuminate how aware parents are of their role in these processes. Parents are clearly gendering their children, but what are the subtleties of the gendered outcomes they seek to construct, why do they seek to construct those, and how aware are they of that construction process?
\item interactionist approaches that view gender as a situated accomplishment: The interactionist approach to gender as accomplishment (West and Fenstermaker 1993, 1995; West and Zimmerman 1987) provides a powerful framework for understanding what I heard about gender nonconformity: view parents not simply as agents of gender socialization but rather as actors involved in a more complex process of accomplishing gender with and for their children. Equally central is the concept of accountability. Accountability is relevant not only when people are doing gender in accordance with the expectations of others but also when they resist or stray from such expectations. 
Accountability is the driving motivator; the specifics of the normative order provide the content, with social interaction the medium” (Fenstermaker and West 2002, 213-14). The “content” provided by the normative order as normative conceptions and view these as historically and locally variable.
\item normative conceptions of masculinity: hegemonic masculinity, a normative conception to which men are accountable, a form of masculinity in relation to which subordinated masculinities, as well as femininities, are defined. Connell (1987, 187) argues that there is no need for a concept of hegemonic femininity, because the fundamental purpose of hegemonic masculinity is to legitimate male domination. He argues that among the features of hegemonic masculinity in this particular time and place are aggression, limited emotionality, and heterosexuality. “ ‘masculinity’ does not exist except in contrast with ‘femininity’ ” (Connell 1995, 68). Connell and Kimmel view homophobia as central to this rejection of femininity.
\end{itemize}

Major focus of the interview questions was on the current activities, toys, clothes, behaviors, and gender awareness of the focal child and the parents’ perceptions of the origins of these outcomes, as well as their feelings about their children’s behaviors and characteristics in relation to gendered expectations. Interviews ended with some general questions about the desirability and feasibility of gender neutrality in childhood. The project focuses on parents’ perceptions and self-reports, and I am not able to compare those to evidence on actual parental behavior

Mothers and fathers, across a variety of social locations, often celebrated what they perceived as gender nonconformity on the part of their young daughters. A few parents combined these positive responses with vague and general negative responses. But these were rare and expressed with little sense of concern, “I wouldn’t want her to be too boyish, because she’s a girl.” But, no parents expressed only negative responses. But it is possible that negative response from parents to perceived departures from traditional femininity would be more notable as girls reach adolescence. Pipher (1998, 286) "That’s when the gender roles get set in cement, and that’s when girls need tremendous support in resisting cultural definitions of femininity."

In stark contrast to the lack of negative response for daughters, 23 of 31 parents of sons expressed at least some negative responses, and 6 of these offered only negative responses regarding what they perceived as gender nonconformity. The most common combination was to indicate both positive and negative responses.

\begin{itemize}
	\item Domestic Skills, Nurturance, and Empathy: Parents accepted, and often even celebrated, their sons’ acquisition of domestic abilities and an orientation toward nurturance and empathy. Encourage domestic competence, nurturance, emotional openness, empathy, and nonviolence as attributes they considered nontraditional but positive for boys. Recent literature on parental coding of toys as masculine, feminine, or neutral, which indicates that parents are increasingly coding kitchens and in some cases dolls as neutral rather than exclusively feminine (Wood, Desmarais, and Gugula 2002). Mothers tended to express them with greater certainty, while fathers were less enthusiastic and more likely to include caveats.
	\item Icons of Femininity: more common were negative responses to items, activities, or attributes that could be considered icons of femininity, consistent with Kimmel’s (1994, 119) previously noted claim that the “notion of anti-femininity lies at the heart of contemporary and historical constructions of manhood" Things like wearing pink or frilly clothing; wearing skirts, dresses, or tights; and playing dress up in any kind of feminine attire, nail polish, dance, especially ballet, and Barbie dolls. Actions they take: “shying away” and “steering”. ‘No, you can’t do that, little girls put nail polish on, little boys don’t.’ "we compromised, we got him a NASCAR Barbie" "he plays with Ken and it doesn’t go much further than that, so I’m fine" "I’d be doing other things to compensate for the fact that I signed him up for dance.” Concern about excessive emotionality (especially frequent crying) and passivity in their sons. Encouraging that son to fight for what he wants
	\item Homosexuality: fear that a son either would be or would be perceived as gay. Not evident in parents’ comments about daughters, nor among gay and lesbian parents, but arose for 7 of the 27 heterosexual parents who were discussing sons. Another possibility: that playing with toys “that are meant for girls” might not indicate but rather shape the son’s eventual sexual orientation. Actions, either actual or hypothetical, taken to discourage homosexuality and accomplish heterosexuality. “If [he] were to be gay, it would not make me happy at all. I would probably see that as a failure as a dad . . ., as a failure because I’m raising him to be a boy, a man”
\end{itemize}

Motivations for the accomplishment of masculinity: Although some parents did speak of their sons as entirely “boyish” and “born that way,” many reported efforts to craft a hegemonic masculinity. Parents indicated their awareness that their sons’ behavior was at risk of gender assessment, an awareness rarely noted with regard to daughters. Parents varied in terms of their expressed motivations for crafting their sons’ masculinity, ranging from a sense of measuring their sons against their own preferences for normative masculinity (more common among heterosexual fathers) to concerns about accountability to gender assessment by peers, other adults, and society in general (more common among heterosexual mothers and gay parents, whether mothers or fathers).

\begin{itemize}
	\item Heterosexual Fathers: The role both Connell and Kimmel argue heterosexual men play in maintaining hegemonic masculinity: passing along that normative conception to their sons may be part of how they accomplish their own masculinity.
	\item Heterosexual Mothers, Lesbian Mothers, and Gay Fathers: express fear for how their sons would be assessed by others if they did not approximate that ideal. “This stupid world cares about what we look like, unfortunately." "I wouldn’t want him to feel out of place” Also notable among the comments expressing accountability to others were reports by gay and lesbian parents who felt under particular scrutiny in relation to their sons’ (but not their daughters’) gender performance. But only heterosexual parents raised fears or concerns about their sons’ eventual sexual orientation. "when you’ve got a husband to deal with it’s harder” taken together with the frequent mentions of male partners’ reactions among heterosexually partnered mothers, it bolsters the contention that accountability to fathers is felt strongly by heterosexual mothers as they assess their sons’ gender performance. Accountability to fathers indicates an indirect path through which heterosexual men may further influence the accomplishment of their sons’ gender.
\end{itemize} 



\newpage 

\section{References}

\begingroup
\bibliographystyle{abbrvnat}
\bibliography{/Users/alexsheridan/Documents/Work/PhD/latex/biblio.bib}
\endgroup
		
	\end{document}